
% this file is called up by thesis.tex
% content in this file will be fed into the main document

%: ----------------------- introduction file header -----------------------
\chapter{Results and conclusions}

\graphicspath{{results/figures/}}

% ----------------------------------------------------------------------
%: ----------------------- introduction content ----------------------- 
% ----------------------------------------------------------------------

\section{Performance}

% table with machine configurations

Performance test were performed on two machines which configuration is presented in table~\ref{tab:test_configuration}. PhysX scene with one emitter was used. Results were rendered in two resolutions - 640~x~480 and 1280~x~960. 

%Present graphs with frame rate depending on number of particle simulated. Note that decrease in performance comes also from simulation cost, not only from rendering.

%Show second set of graphs showing time to render one frame depending on number of particles. 

%Show table showing how resolution affects performance of algorithms when number of particles is kept constant.

%Show table showing speedup of isosurface algorithm.

\begin{table}   
    \caption[Configuration of test machines]{\textbf{Configuration of test machines}}
    \centering
    \begin{tabular}{ l | p{4.5cm} | p{4.5cm} }
                                         & machine 1                                             & machine 2 \\ \hline
        CPU                           & Intel Core i7 2600k 3.4 GHz, 4 cores  & Inte Core i7 2630M 2.0 GHz, 4 cores \\ \hline
        GPU                           & Nvidia GeForce GTX 560 Titanium        &  Nvidia GeForce GT 540M \\ \hline
        Physical memory       & 8 GB                                                      & 6GB \\
    \end{tabular}
    \label{tab:test_configuration}
\end{table}

\begin{table}
   \caption[Performance comparison]{\textbf{Performance comparison} of all rendering algorithms. Screen space methods ran with 60000 particles, isosurface extraction run with 20000 particles using 3 threads.}
   \centering
   \begin{tabular} { l | r | c | r }
      Algorithm                                       &            resolution              &          machine           &          Frame (ms)      \\ \hline \hline
      Bilateral Gaussian smoothing        &           640x480                &              1                  &           10.0                    \\ \hline
      Bilateral Gaussian smoothing        &           1280x960              &              1                  &           13.8                    \\ \hline
      Bilateral Gaussian smoothing        &           640x480                &              2                  &           51.9                    \\ \hline
      Bilateral Gaussian smoothing        &           1280x960              &              2                  &           63.5                    \\ \hline \hline

      Curvature flow smoothing 100 it. &           640x480                &              1                  &           15.6                    \\ \hline
      Curvature flow smoothing 100 it. &          1280x960               &              1                  &           27.8                    \\ \hline
      Curvature flow smoothing 100 it. &           640x480                &              2                  &           73.1                    \\ \hline
      Curvature flow smoothing 100 it. &          1280x960               &              2                  &           134.1                  \\ \hline \hline

      Curvature flow smoothing 60 it.   &           640x480                &              1                  &           13.7                \\ \hline
      Curvature flow smoothing 60 it.   &          1280x960               &              1                  &           22.0                \\ \hline
      Curvature flow smoothing 60 it.   &           640x480                &              2                  &           65.7                    \\ \hline
      Curvature flow smoothing 60 it.   &          1280x960               &              2                  &           104.5               \\ \hline \hline

      Isosurface extraction                    &          640x480                &              1                  &            60.4                    \\ \hline     
      Isosurface extraction                    &          1280x960               &              1                  &           63.1                   \\ \hline      
      Isosurface extraction                    &          640x480                &              2                  &            104.8               \\ \hline
      Isosurface extraction                    &          1280x960               &              2                  &           120.5               \\ 
   \end{tabular}
\end{table}

\figuremacro{curvature_640_480}{Performance of screen space rendering with curvature flow at 640x480}{Some description}
\figuremacro{curvature_1280_960}{Performance of screen space rendering with curvature flow at 1280x960}{Some description}
\figuremacro{bilateral}{Performance of screen space rendering with bilateral Gaussian smoothing}{Some description}
\figuremacro{curvature_100it}{Performance of screen space rendering curvature flow smoothing}{100 iterations were performed for each frame.}

%CONCLUSIONS:
%- increasing resolution affects screen space rendering much more than increasing number of particles
%  - curvature flow affected much more by increasing resolution than Gaussian smoothing
%- isosurface extraction behaves in opposite way

\section{Visual appearance}

Describe what features algorithms include. Show some images. 

- isosurface extraction looks good but it's really slow, loads CPU and does not include depth.
- curvature flow produces better smoothed surface but it comes with a cost of greater complexity (computational)
- isosurface produces fluids that is more gelly like than screen space rendering


FINAL CONCLUSIONS:
- Isosurface is sufficient for small particle systems (up to ~3k - 6k particles) like blood, small fountains.
- Screen space rendering can be used to simulate bigger systems - water flooding some areas, ponds. 
   - can run bilateral gauss on lower performance hardware and curvature flow on higher.


% ----------------------------------------------------------------------



