
% Thesis Abstract -----------------------------------------------------


%\begin{abstractslong}    %uncommenting this line, gives a different abstract heading
\begin{abstracts}        %this creates the heading for the abstract page

Computing power growth allows more complex and accurate simulation techniques to be implemented in 
real-time applications. An example of those techniques are particle-based methods for simulation of fluids. 
They can provide a new level of realism into computer games. Nowadays most fluids in games are simulated using height field. 
Although using this method realistic waterbodies (like oceans or ponds) can be simulated, it's hard to achieve effects such as splashing or flooding.
Those can be easily simulated with particle-based methods such as Smoothed Particle Hydrodynamics (SPH). 

In this paper I would like to present realistic model of fluid that can be used in real time application, especially in computer games. Emphais will be placed on realistic rendering output from particle simulation.
For water simulation NVIDIA PhysX SDK will be used. For rendering particles I will use two different approaches: screen space rendering and isosurface extraction. Second algorithm is described in \cite{RosenbergBirdwell2008} and is running on cpu. Authors presented performance of this algorithm running on one CPU core and I will present multithread implementation and it's performance analysis. At the end of this paper I will also present comparision of those two rendering techniques. 


\end{abstracts}
%\end{abstractlongs}


% ---------------------------------------------------------------------- 
