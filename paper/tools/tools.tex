
% this file is called up by thesis.tex
% content in this file will be fed into the main document

%: ----------------------- introduction file header -----------------------
\chapter{Tools}


\graphicspath{{tools/figures/}}

% ----------------------------------------------------------------------
%: ----------------------- introduction content ----------------------- 
% ----------------------------------------------------------------------

This chapter will present brief introduction of tools used in building visualization system. This is no intended to be a tutorial as those information can be found on the Internet. 

\section{Opengl}
OpenGL is an API specification for interacting with graphic hardware in order to produce 2D and 3D graphic. 
OpenGl is an API specification for creating 2D and 3D graphics using dedicated graphic hardware (although software implementation of OpenGL API exists, they are really slow). As a specification it is not bound to any language, operating system or hardware. Nowadays most of (if not all) graphic cards support some version of OpenGL API. It consists of core functionality and extensions. Core functionality are functions that have to be supported and extensions as name indicates are extra features which cane either increase performance or make programming easier.

At first OpenGL provided only fixed graphic pipeline functionality - which means programmers couldn't interfere in how things were done on hardware. As hardware developed such a functionality was added and coexisted with fixed one. However fixed pipeline functionality is good for learning for professional usage it's better to have a full control over the pipeline. So with introduction of OpenGL 3.0 fixed pipeline functionality became obsolete and 

TODO core profile and compatibility profile

TODO reference to opengl bindings http://www.opengl.org/resources/bindings/

TODO reference to opengl superbible

TODO image of graphic pipeline

TODO mention about OpenGL ES

\section{boost}
Boost is a set of libraries extending capabilities of C++ language. Libraries provides classes and functions for most common task and algorithms, like reqular expressions, hash maps, threading. They are also cross platform. 

The advantage of boost libraries is that they are designed for maximum flexibility and speed. They make heavy use of C++ template programming to be as general as possible. The other advantage of boost libraries is that they often become included in C++ standard - like regular expressions (FIXME). 

On the other hand relying on templates makes compiling times longer. Template classes and functions must be defined entirely in header files what makes it impossible to compile them into object files. Thus whenever file has to be recompiled template classes have to be generated over again.

Boost libraries used in my project are boost regex and boost threads. The first one is used to process configuration files. Boost trheads library is used to parallelize isosurface extraction algorithm presented in (TODO reference).

TODO reference to documentation or main page

\section{PhysX}

PhysX is a framework for performing physical simulations in real time. It's designed to be used in computer games and optimized for performance. It provides most necessary models required in games: rigid bodies, soft bodies, cloths, fluids and joints. What is more it allows simulations to be performed on GPU which is much faster especially when large amounts of object acts in simulations.

The main drawback is that hardware acceleration can only be performed on Nvidia GPUs. 

PhysX provides fluid simulations with SPH method which is used in this project. 

TODO prezentacja
% ----------------------------------------------------------------------



